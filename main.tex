\documentclass{article}

% Language setting
% Replace `english' with e.g. `spanish' to change the document language
\usepackage[english]{babel}

% Set page size and margins
% Replace `letterpaper' with `a4paper' for UK/EU standard size
\usepackage[a4paper,top=2cm,bottom=2cm,left=3cm,right=3cm,marginparwidth=1.75cm]{geometry}

% Useful packages
\usepackage{amsmath}
\DeclareMathOperator{\sinc}{sinc}
\DeclareMathOperator*{\argmax}{arg\,max}
\DeclareMathOperator*{\argmin}{arg\,min}

\usepackage{amssymb}
\usepackage{float}
\usepackage{graphicx}
\usepackage[colorlinks=true, allcolors=blue]{hyperref}

\usepackage{tikz}
\usetikzlibrary{arrows,decorations.pathmorphing,backgrounds,fit,positioning,shapes.symbols,chains,shapes.geometric,shapes.arrows,calc}

\usepackage{listings}
\usepackage{xcolor}
\usepackage{enumitem}
\usepackage{physics}
\setlist[itemize]{nosep} 

\definecolor{codegreen}{rgb}{0,0.6,0}
\definecolor{codegray}{rgb}{0.5,0.5,0.5}
\definecolor{codepurple}{rgb}{0.58,0,0.82}
\definecolor{backcolour}{rgb}{0.95,0.95,0.92}

\lstdefinestyle{mystyle}{
  backgroundcolor=\color{backcolour},   
  commentstyle=\color{codegreen},
  keywordstyle=\color{magenta},
  numberstyle=\tiny\color{codegray},
  stringstyle=\color{codepurple},
  basicstyle=\ttfamily\footnotesize,
  breakatwhitespace=false,         
  breaklines=true,                 
  captionpos=b,                    
  keepspaces=true,                 
  numbers=left,                    
  numbersep=5pt,                  
  showspaces=false,                
  showstringspaces=false,
  showtabs=false,                  
  tabsize=2
}

\lstset{style=mystyle}

\title{Computer Vision}
\author{Matteo Galiazzo}

\begin{document}

\maketitle

\tableofcontents

\section{Introduction}

The difference between computer vision and image processing is the fact that computer vision is the process of extracting information from images, while image processing aims at improving the quality of images.
Quite often image processing helps computer vision.
The informations we want to extract from the images could be counting, object orientation, object classification, measurements...
Computer vision is challenging since we lose depth from the images (3D information becomes 2D), scale and illumination varies, and there's object occlusion (object hiding other objects).

Computer vision started with hand-crafted decision rules that only required few images as example, and evolved to machine learning where the algorithm learns a decision rule, but the training requires hundreds to thousands of images.
The big paradigm shift happened with deep learning (e2e learning) which learned both the image representation and the decision rules, but required thousands to millions of words.
Deep learning has been enabled by better networks, better hardware and more data.


\section{Fundamentals of Image Processing and Computer Vision}

\section{Advanced Topics in Deep Learning for Computer Vision}

% \begin{figure}[htbp]
%   \centering
%   \includegraphics[width=0.8\linewidth]{./img/rnn_unfolded.png}
%   \caption{unfolded RNN}
%   % \label{}
% \end{figure}

\end{document}
